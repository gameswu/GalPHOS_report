\documentclass[12pt]{article}
\usepackage{fancyhdr} % 用于自定义页眉页脚
\usepackage{listings} % 用于代码高亮
\usepackage{xcolor} % 用于定义颜色
\usepackage{fontspec} % 用于设置字体 (需要XeLaTeX或LuaLaTeX)
\usepackage{graphicx} % 用于插入图片
\usepackage{float} % 用于强制图片位置
\usepackage{geometry} % 用于设置页面边距

\geometry{a4paper, margin=1in} % 设置A4纸大小和1英寸边距

% 设置图片路径
\graphicspath{{./fig/}} % 假设图片存放在images文件夹中

% 定义VS Code风格的颜色
\definecolor{vscode-bg}{RGB}{30,30,30}
\definecolor{vscode-string}{RGB}{206,145,120}
\definecolor{vscode-keyword}{RGB}{86,156,214}
\definecolor{vscode-comment}{RGB}{106,153,85}
\definecolor{vscode-number}{RGB}{181,206,168}
\definecolor{vscode-function}{RGB}{220,220,170}

% 设置代码块字体(Fira Code或Consolas或Monaco)
\setmonofont{Monaco}[Scale=0.9]

% 配置listings样式
\lstdefinestyle{vscode}{
    backgroundcolor=\color{vscode-bg},
    commentstyle=\color{vscode-comment},
    keywordstyle=\color{vscode-keyword}\bfseries,
    numberstyle=\tiny\color{white},
    stringstyle=\color{vscode-string},
    basicstyle=\ttfamily\footnotesize\color{white},
    breakatwhitespace=false,
    breaklines=true,
    captionpos=b,
    keepspaces=true,
    numbers=left,
    numbersep=5pt,
    showspaces=false,
    showstringspaces=false,
    showtabs=false,
    tabsize=2,
    frame=single,
    frameround=tttt,
    rulecolor=\color{gray!30}
}

% 设置默认代码样式
\lstset{style=vscode}

\title{GalPHOS Project Report}
\author{\textbf{Group 16}: Hengxu Wu, Kairui Li, Yangyi Xiong}
\date{}

% 设置页面样式
\pagestyle{fancy}
\fancyhf{} % 清除所有页眉页脚
\fancyhead[L]{\leftmark} % 左上方显示section
\fancyhead[R]{\rightmark} % 右上方显示subsection
\fancyfoot[C]{\thepage} % 页脚中央显示页码

% 重新定义section和subsection的标记格式
\renewcommand{\sectionmark}[1]{\markboth{#1}{}}
\renewcommand{\subsectionmark}[1]{\markright{#1}}

\begin{document}
\maketitle

\section{Introduction}
\subsection{Project Overview}
% 简单介绍项目的内容和情况

GalPHOS (Galaxy Physics Online System) is a physical competition event management system based on the web platform, covering a variety of functions including event registration, exam management, and score statistics. The system is designed to provide an efficient and convenient management platform for physics competitions, supporting a variety of user roles, including administrators, students, coaches and graders.

Our system architecture and functions are based on the life cycle design of a competition, including five stages: registration, pre-match preparation, answer submission, post-match marking, result statistics and archives. The figure below illustrates the workflow of the system.
\begin{figure}[H]
    \centering
    \includegraphics[width=\textwidth]{Fig1.png}
    \caption{GalPHOS System Workflow}
    \label{fig:workflow}
\end{figure}

\subsection{Our Work}
% 介绍小组成员的工作内容和分工

Our group consists of three members: Hengxu Wu, Kairui Li, and Yangyi Xiong. The leader of the group is Hengxu Wu. Each member has contributed to different aspects of the project:
\begin{itemize}
    \item Hengxu Wu: Responsible for project design, development documents, and front-end development.
    \item Kairui Li: % TODO
    \item Yangyi Xiong: % TODO
\end{itemize}

\section{System Architecture}
\subsection{Architecture Overview}
% 描述系统的整体架构和设计思路,包括前端和后端的分离
\subsection{Frontend Design}
% 介绍前端的设计和实现(类型这一块)
\subsection{Backend Design}
% 介绍后端的设计和实现(类型这一块)

\section{Usage and Features}
\subsection{Administrator}
\subsubsection{Admin Management}
We have a super administrator account initialized along with the system, which has username and password both initialized but can be changed later. All other administrators are added by
the super administrator later, and super admin can manage them by adding, deleting, banning and password reseting.
\begin{figure}[htbp] % figure环境,htbp参数指定图片位置
    \centering % 图片居中显示
    \includegraphics[width=0.6\textwidth]{./fig/super admin adminmanage.png} % 插入图片,设置宽度为文本宽度的60%
    \caption{super admin management} % 添加图片标题
    \label{fig:example} % 添加标签,用于引用
\end{figure}
Except the admin management
function, the other administrators are able to manage the system in the same way as the super administrator.
\subsubsection{Dashboard}
The admins' dashboard shows information about the system, including the number of registers, number of active users, number of provinces and schools,
information about the exams and so on.
\subsubsection{User Management}
Administrators can check and manage users in the user management page: administrators can access the detailed information of
the users, including their username, phone number, region, status and other information,
for registering users, administrators can approve or reject their registration requests, and for existing users, administrators can ban or unban them,
and delete them if necessary.
The user management page also provides a search function, allowing administrators to search for users by username, phone number, region,
status and other criteria.
\subsubsection{Region Management}
In region management page, administrators can access and manage the active regions and schools in the system. In the 'Province management'
part, administrators can add, delete provinces and add schools to provinces. And in the 'School management' part, administrators can
access the detailed information of the schools, including their name and belonging region.
\subsubsection{Region Changing Management}
In realistic life, some students may change their region or school after registration, so the region changing management page is designed to
manage the region changing requests from students. After the student submits a region changing request,
the administrators can check the request and approve or reject it. If approved, the student's region will be updated accordingly.
This page provides a dashboard showing the number of requests, and a table showing the detailed information of the requests
and allowing administrators to work on the requests on it.
\subsubsection{Exam Management}
The exams are managed by the administrators in the exam management page. Administrators can access the information of the exams, or create, delete and modify exams here.
The dashboard shows the number of exams, and the table shows the detailed information of the exams, including their name, start time, end time, status and other information and
provides a filter function to filter the exams by status.
The exams have total five states: 'Not Published', 'Published', 'In Progress', 'Finished' and 'Graded'.
To create an exam, administrators need to provide basic exam information: title, detailed information, number of questions, total score,
time limit, start time and end time, and the files of test paper, answer sheet and solution. We also provide the function to save
current exam as a draft with the state 'Not Published', so that administrators can modify it later.
After the exam is published, other users can see the exam in their pages, and the exam will automatically change its state to 'In Progress' when the start time is reached and
end its state to 'Finished' when the end time is reached. Only when in the 'In Progress' state, students can submit their answer.
In the 'Finished' state, the answer sheets can be distributed to the graders by the administrator and the graders can grade the answer sheets in the 'Graded' state.
After all answer sheets are graded, the exam will automatically change its state to 'Graded', and the results will be calculated and stored in the database
so that users can check their results later.
\subsubsectin{Grading Management}
After a exam is finished, the administrators can distribute the answer sheets to the graders in the grading management page.
In this page, administrators can check the current status of the system, including the number of answer sheets to be graded, the number of graders,
and the progress of grading. The administrators can assign different questions to different graders, while the grading service
will automatically assign the questions to the graders based on their availability and workload.
\subsection{Student}
\subsubsection{Dashboard}
The students' dashboard shows the information about the ongoing and finished exams, including the grades of the finished exams and the status of the ongoing exams.
\subsubsection{Ongoing Exams}
The ongoing exams page shows the exams that are currently in progress, and students can access the detailed information,
download files and submit their answers here.
\subsubsection{Finished Exams}
This page shows the exams that are finished, and students can check their grades and the solutions of the exams here.
We also provide the function to download the xlsx file of the results, including the detailed grades, ranking in the region, and the overall ranking.
\subsubsection{Account Management}
We allow students to manage their accounts, including changing their passwords, updating their avatar, and updating their username.
\subsection{Coach}
\subsubsection{Dashboard}
In the coaches' dashboard, coaches can check the information about their students, including the number of (active) students, ongoing/finished exams,
average grades and so on.
\subsubsection{Student Management}
We provide the function for coaches to manage their students, including adding, deleting and editing students.
In our system, students belonging to a coach themselves have no accounts, and all their operations are done by the coach.
\subsubsection{Ongoing Exams}
Students belonging to a coach have no accounts, so they cannot access the exams directly, instead, coaches can access the exams
for them: the coach can download the files, and submit the answers for the students.
\subsubsection{History Exams}
This page shows the finished exams of the students just like the students' finished exams page, and coaches can check the grades and solutions of the exams
for all their students.
\subsubsection{Account Management}
We allow coaches to manage their accounts, including changing their passwords, updating their avatar, and updating their username.
\subsection{Grader}
\subsubsection{Dashboard}
In the graders' dashboard, graders can check the information about the exams they are assigned to grade, including the number of answer sheets to be graded or finished,
grading process information, and so on.
\subsubsection{Grading Queue}
In the grading queue page, graders can choose the test they want to grade now and grade their assignments.
\subsubsection{Account Management}
We allow graders to manage their accounts, including changing their passwords, updating their avatar, and updating their username.

\section{Summary}
% 对项目的总结和展望
\end{document}